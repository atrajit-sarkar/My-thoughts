% !TeX encoding = UTF-8 Unicode
\documentclass[12pt,a4paper]{article}


% Page layout
\usepackage{geometry}
\usepackage{amsmath}
\usepackage{amssymb}
\geometry{margin=1in}

% Math packages
\usepackage{amsmath}
\usepackage{amssymb}
\usepackage{amsfonts}
\usepackage{mathtools}
\usepackage{amsthm}
\usepackage{bm}
\usepackage{mathrsfs}

% Useful packages
\usepackage{graphicx}
\usepackage{hyperref}
\usepackage{enumitem}
\usepackage{tikz-cd}

% Theorem environments
\newtheorem{theorem}{Theorem}[subsection]
\newtheorem{lemma}[theorem]{Lemma}
\newtheorem{proposition}[theorem]{Proposition}
\newtheorem{corollary}[theorem]{Corollary}
\newtheorem{example}[theorem]{Example}
\theoremstyle{definition}
\newtheorem{definition}[theorem]{Definition}
\theoremstyle{remark}
\newtheorem{remark}[theorem]{Remark}

% New commands
% \newcommand{\properideal}{\underset{\neq}{\vartriangleleft}}

% Title
\title{Elementary Number Theory}
\author{Atrajit Sarkar}
\date{\today}

\begin{document}

\maketitle

\begin{abstract}
    Hello
\end{abstract}
\tableofcontents

\pagebreak

\section{Primitive Roots}
\subsubsection{primitive roots of $p^2$}
We have primitive root of $p$, let that be $r$.  Then $r^{p-1}\equiv 1 \mod p$. Now we have if $r^{p-1}\not\equiv 1 \mod p^2$ then $r$ is the primitive root of $p^2$. Then we have total number of primitive roots are $\phi(\phi(p^2))=(p-1)\phi(p-1)$. Our goal here is to find explicitly what are they.

\textbf{Claim:} If $r^{p-1}\equiv 1 \mod p^2$ then we have for $r'=r+kp \quad \forall k=1(1)(p-1)$, $(r')^{p-1}\not \equiv 1 \mod p^2$ and hence we have in total $p-1$ many incongruent primitive roots of $p^2$ for each $r$ with this property. 

\textbf{Claim:} For $r^{p-1} \not \equiv 1 \mod p^2$ we already have it to be a primitive root. Then considering the set $\{r+kp | \quad 0<k<p\}$. There exists exactly one elelemnt in this set such that $(r+kp)^{p-1}\equiv 1 \mod p^2$. Hence we get exactly $p-1$ primitive roots. So intotal $(p-1)\phi(p-1)$ many. And hence they are the exact primitive roots of $p^2$. 

Now, we are going to find the exact form of $k$ for which $(r+kp)^{p-1}\equiv 1 \mod p^2$ so that we can easily find out it and exclude it from primitive roots set. 

Note that $r^{p-1} \not \equiv 1 \mod p^2$ in this case and $r^{p-1}\equiv 1 \mod p$ that means $r^{p-1} =1+pk_1+p^2k_2$, where $p \nmid k_1$. $(r+kp)^{p-1}\equiv r^{p-1}+kp(p-1)r^{p-2} \equiv 1+pk_1-kpr^{p-2} \mod p^2$. Now, for $k \equiv k_1r \mod p$, we have the desired result. Now, as $k_1,r$ is unique for each $r$ hence is $k$ hence we exclude  only one member.
\end{document}