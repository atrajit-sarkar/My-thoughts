% !TeX encoding = UTF-8 Unicode
\documentclass[12pt,a4paper]{article}


% Page layout
\usepackage{geometry}
\usepackage{amsmath}
\usepackage{amssymb}
\geometry{margin=1in}

% Math packages
\usepackage{amsmath}
\usepackage{amssymb}
\usepackage{amsfonts}
\usepackage{mathtools}
\usepackage{amsthm}
\usepackage{bm}
\usepackage{mathrsfs}

% Useful packages
\usepackage{graphicx}
\usepackage{hyperref}
\usepackage{enumitem}
\usepackage{tikz-cd}
\usepackage[backend=biber,style=ieee]{biblatex}
\addbibresource{elementary-number-theory-david-M-Burton.bib}


% Theorem environments
\newtheorem{theorem}{Theorem}[subsection]
\newtheorem{lemma}[theorem]{Lemma}
\newtheorem{proposition}[theorem]{Proposition}
\newtheorem{corollary}[theorem]{Corollary}
\newtheorem{example}[theorem]{Example}
\theoremstyle{definition}
\newtheorem{definition}[theorem]{Definition}
\theoremstyle{remark}
\newtheorem{remark}[theorem]{Remark}

% New commands
% \newcommand{\properideal}{\underset{\neq}{\vartriangleleft}}

% Title
\title{Elementary Number Theory}
\author{Atrajit Sarkar}
\date{\today}

\begin{document}

\maketitle

\begin{abstract}
    In this article I want to share some of my thoughts that I found interesting during my study on Elementary Number Theory by David M. Burton \cite{DMB7} and other referrences that is included in the referrence section.
\end{abstract}
\tableofcontents

\pagebreak

\section{Finding Primitive Roots}
\subsection{Finding for $p^2$}
We have primitive root of $p$, let that be $r$.  Then $r^{p-1}\equiv 1 \mod p$. Now we have if $r^{p-1}\not\equiv 1 \mod p^2$ then $r$ is the primitive root of $p^2$. Then we have total number of primitive roots are $\phi(\phi(p^2))=(p-1)\phi(p-1)$. Our goal here is to find explicitly what are they.

\textbf{Claim:} If $r^{p-1}\equiv 1 \mod p^2$ then we have for $r'=r+kp \quad \forall k=1(1)(p-1)$, $(r')^{p-1}\not \equiv 1 \mod p^2$ and hence we have in total $p-1$ many incongruent primitive roots of $p^2$ for each $r$ with this property. 

\textbf{Claim:} For $r^{p-1} \not \equiv 1 \mod p^2$ we already have it to be a primitive root. Then considering the set $\{r+kp \text{ }| \quad 0<k<p\}$. There exists exactly one elelemnt in this set such that $(r+kp)^{p-1}\equiv 1 \mod p^2$. Hence we get exactly $p-1$ primitive roots. So intotal $(p-1)\phi(p-1)$ many. And hence they are the exact primitive roots of $p^2$. 

Now, we are going to find the exact form of $k$ for which $(r+kp)^{p-1}\equiv 1 \mod p^2$ so that we can easily find out it and exclude it from primitive roots set. 

Note that $r^{p-1} \not \equiv 1 \mod p^2$ in this case and $r^{p-1}\equiv 1 \mod p$ that means $r^{p-1} =1+pk_1+p^2k_2$, where $p \nmid k_1$. $(r+kp)^{p-1}\equiv r^{p-1}+kp(p-1)r^{p-2} \equiv 1+pk_1-kpr^{p-2} \mod p^2$. Now, for $k \equiv k_1r \mod p$, we have the desired result. Now, as $k_1,r$ is unique for each $r$ hence is $k$ hence we exclude  only one member.

\subsection{Finding for $p^k$}
Now we use the idea got in the above section. So, we have the following lemma.

\begin{lemma}
    If $r$ is a primitive root of $p^2$ then $r^{p^{k-2}(p-1)}\not \equiv 1 \mod p^k$.
\end{lemma}
and we have the following corollary

\begin{corollary}
    If $r$ is a primitive root of $p^2$ then $(r+kp^2)^{p^{k-2}(p-1)}\not \equiv 1 \mod p^{k}$ for all $k=1,\cdots p^{k-2}-1$ and hence we have for each primitive root of $p^2$, $p^{k-2}$ many primitive roots for $p^{k}$.
\end{corollary}
 Using lemma1.2.1 and corollary 1.2.2 we have total number of primitive roots for $p^{k}$ is $p^{k-2}(p-1)\phi(p-1)=\phi(\phi(p^k))$. And hence these elemets are explicitly all the primitive roots of $p^k$.

 \subsection{Finding for $2p^k$}
\begin{lemma}
    If $r$ is a primitive root of $p^k$ so is for $2p^k$.
\end{lemma}

Using this lemma 1.3.1 and the fact that $\phi(2p^k)=\phi(p^k)$ we have all the pimitive roots of $p^k$ are exactly the primitive roots of $2p^k$.

\subsection{Summary}
Now all together if we are given any composite number $n$ and we are to find the primitive root of it we just need to check its form. Suppose $n=2p^k$ then we find the primitive root of it in the followign steps:

\begin{enumerate}
    \item First find one primtive root of $p$ using trial and error method. Just we need to check all the devisors of $p-1$ upto $p-1/2$. After getting one such say $r$. Then we get the other primtive roots as $\{r^m \text{ }| \quad gcd(m,p-1)=1\}$.
    \item Now we are going to find primitive roots of $p^2$. For that use the subsection 1.1.
    \item Now we are going to find primitive roots of $p^k$. To do that use subsection 1.2. This step is easiest one after finding all the primitive roots of $p^2$.
    \item Now finally these are explicitly the primitive roots of $2p^k \mod 2p^k$ by subsection 1.3.
\end{enumerate}



\printbibliography




\end{document}